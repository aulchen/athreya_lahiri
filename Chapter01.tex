\documentclass{article}
\usepackage{amsmath}
\usepackage{amsfonts}
\usepackage{amssymb}
\usepackage{optidef}

\newenvironment{proof}{\paragraph{Proof:}}{\hfill$\square$}
\newtheorem{theorem}{Theorem}
\newtheorem{lemma}[theorem]{Lemma}
\newtheorem{corollary}[theorem]{Corollary}

\usepackage{graphicx}
\graphicspath{ {images} }

\newcommand{\R}{\mathbb{R}}
\newcommand{\Q}{\mathbb{Q}}
\newcommand{\Z}{\mathbb{Z}}
\newcommand{\N}{\mathbb{N}}

\newcommand{\F}{\mathcal{F}}

	
\newcommand{\indep}{\perp \!\!\! \perp}

\newcommand{\prob}{\boldsymbol{P}}
\newcommand{\E}{\text{E}}
\newcommand{\var}{\text{Var}}
\newcommand{\cov}{\text{Cov}}
\newcommand{\tr}{\text{tr}}
\newcommand{\sd}{\text{SD}}
\newcommand{\se}{\text{SE}}
\newcommand{\dom}{\text{dom}}
\newcommand{\quartile}{\text{quartile}}

\newcommand{\trace}{\text{trace}}

\newcommand{\bin}{\text{Bin}}
\newcommand{\pois}{\text{Pois}}
\newcommand{\geom}{\text{Geom}}

\author{Arthur Chen}
\title{Athreya Lahiri Chapter 1 Solutions}
\date{\today}

\begin{document}

\maketitle

\section*{Problem 1.19}

Let $\Omega$ be a nonempty set and let $C \subset \mathcal{P} (\Omega)$ be a semialgebra. Let

\[
\mathcal{A}(C) = \{
A: A=\bigcup_{i=1}^k B_i:B_i \in C, i=1,2 \dots k, k\in \N
\}
\]

\subsection*{Part a}

Show that $\mathcal{A}(C)$ is the smallest algebra containing $C$.

\begin{lemma}
$\mathcal{A}(C)$ is an algebra.
\begin{proof}
For $\Omega \in \mathcal{A}(C)$, let $A\in C$ be arbitrary. Because $C$ is a semialgebra, $A^C = \bigcap_{i=1}^k B_i, B_i \in C$, so $A^C \in C$. Thus $A \cap A^C = \Omega \in \mathcal{A}(C)$.

For closure under compliments, $A \in \mathcal{A}(C)$ implies by definition that $A = \bigcup_{i=1}^k B_i, B_i \in C$, and taking compliments $A^C = \bigcap_{i=1}^k B_i^C$. Because $B_i \in C$ and $C$ is a semialgebra, each $B_i^C$ is the finite union of $C_j \in C$. Thus
\[
A^C = \bigcap_{i=1}^k\bigcup_{j=1}^{l_i}C_j
\]
Distributing the intersection, we get the union of a finite number of pairwise intersections, e.g. $A^C = (C_{11}\cap C_{12})\cup(C_{11}\cap C_{13})\cup \dots$. All of the pairwise intersections are in $C$, and thus their union is in $\mathcal{A}(C)$. Thus $A^C \in \mathcal{A}(C)$.

Closure under finite union is immediate. 
\end{proof}
\end{lemma}

Showing that $\mathcal{A}(C)$ is the smallest algebra containing $C$ is equivalent to showing that if $B$ is an algebra such that $C \subset B$, then $\mathcal{A}(C) \subset B$. But this is almost immediate. Let $M \in \mathcal{A}(C)$. By definition of $\mathcal{A}(C)$, $M = \bigcap_{i=1}^k M_k, M_k \in C, k \in \N$. Since $B$ is an algebra contain $C$, $M \in B$. Thus $\mathcal{A}(C) \subset B$, as desired.

\subsection*{Part b}

Show that $\sigma\langle C\rangle = \sigma\langle \mathcal{A}(C) \rangle$.

Trivially, $C \subset \mathcal{A}(C)$ implies $\sigma\langle C\rangle \subset \sigma\langle \mathcal{A}(C) \rangle$. For the reverse, let $M \in \sigma\langle \mathcal{A}(C) \rangle$. Then $M$ is the countable union of elements in $\mathcal{A}(C)$, and thus it is the countable union of elements in $C$ and thus in $\sigma\langle C\rangle$.

\section*{Problem 1.20}

Let $C$ be a semialgebra $\Omega$ with $\mu$ a measure on $C$. Let $\mu^*$ be the outer measure induced by $\mu^*$, defined as

\[
\mu^*(A) \coloneq \inf\Biggl\{
\sum_{n=1}^\infty \mu(A_n):\{A_n\}_{n\geq 1} \subset C, A \bigcup_{n\geq 1} A_n
\Biggr\}
\]

Show that $\mu^*$ satisfies the following three properties making it an outer measure:
\begin{itemize}
\item Non-negativity: $\mu^*(\emptyset) = 0$
\item Monotonicity: $A \subset B \Rightarrow \mu^*(A) \leq \mu^*(B)$
\item Countable sub-additivity: For any $\{A_n\}_{n \geq 1} \subset P(\Omega), \mu^*\left( \bigcup_{n\geq 1} A_n \right) \leq \sum_{n=1}^\infty \mu^*(A_n)$
\end{itemize}

\begin{lemma}
For a (non-empty) semi-algebra $C$, $\emptyset \in C$.
\begin{proof}
Let $A \in C$. Then $A^C = \bigcup_{i=1}^k B_i$, $k$ finite, $B_i \in C$. Then $A \cap B_i = \emptyset$, and since $C$ is closed to complements, $\emptyset \in C$.
\end{proof}
\end{lemma}

For non-negativity, $\emptyset \in C$ and $\mu(\emptyset) = 0$ imply $\mu^*(\emptyset) = 0$. For monotonicity, consider any cover of $B$ with sets in $C$. Then that cover also covers $A$. Since $\mu^*(A)$ is the infimum of the sum of the measures of sets in $C$ that cover $A$, monotonicity follows immediately.

For countable sub-additivity, if the sum on the right is infinite, then the result is immediate. Otherwise, $\mu^*(A_n) < \infty$ for all $n$ and the outer measures of $A_n$ decrease enough so that the sum is (absolutely) convergent.

Let $0 < \epsilon < \infty$. Since $\mu^*(A_n)$ is the infimum over covers of $A_n$ with elements in $C$, there exist $\{A_{nj}\}_{j \geq 1} \subset C$ such that

\[
\mu^*(A_n) \leq \sum_{j=1}^\infty \mu(A_{nj}) \leq \mu^*(A_n) + \frac{\epsilon}{2^n}
\]

The union of covers over the $A_n$, $\bigcup_{n \geq 1} \bigcup_{j \geq 1} A_{nj,}$ form a cover for $\bigcup_{n \geq 1} A_n$. Thus

\[
\mu^*\left( \bigcup_{n \geq 1} A_n \right) \leq
\mu\left( \bigcup_{n=1}\bigcup_{j=1} A_{nj} \right)
\leq \sum_{n=1}\sum_{j=1} \mu(A_{nj}) \leq \sum_{n=1}\left(\mu^*(A_n) + \frac{\epsilon}{2^n}\right)
\]

which proves the desired result.

\end{document}