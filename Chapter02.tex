\documentclass{article}
\usepackage{amsmath}
\usepackage{amsfonts}
\usepackage{amssymb}
\usepackage{optidef}
\usepackage{dsfont}

\newenvironment{proof}{\paragraph{Proof:}}{\hfill$\square$}
\newtheorem{theorem}{Theorem}
\newtheorem{lemma}[theorem]{Lemma}
\newtheorem{corollary}[theorem]{Corollary}

\newcommand{\R}{\mathbb{R}}
\newcommand{\Q}{\mathbb{Q}}
\newcommand{\Z}{\mathbb{Z}}
\newcommand{\N}{\mathbb{N}}

\newcommand{\F}{\mathcal{F}}
\newcommand{\M}{\mathcal{M}}
\newcommand{\B}{\mathcal{B}}
	
\newcommand{\indep}{\perp \!\!\! \perp}

\newcommand{\prob}{\boldsymbol{P}}

\author{Arthur Chen}
\title{Athreya Lahiri Chapter 2 Solutions}
\date{\today}

\begin{document}

\maketitle

\section*{Problem 2.1}

Prove de Morgan's laws. Let $\Omega_i, i = 1, 2$ be two nonempty sets, and let $T:\Omega_1 \rightarrow \Omega_2$ be a map. For any collection $\{A_\alpha: \alpha \in I\}$ of subsets of $\Omega_2$, prove that

\begin{gather*}
T^{-1}\left( \bigcup_{\alpha \in I} A_\alpha \right) = \bigcup_{\alpha \in I} T^{-1}(A_\alpha) \\
T^{-1}\left( \bigcap_{\alpha \in I} A_\alpha \right) = \bigcap_{\alpha \in I} T^{-1}(A_\alpha) \\
\left( T^{-1}(A) \right)^C = T^{-1}(A^C)
\end{gather*}

For the first,
\begin{align*}
T^{-1}\left( \bigcup_{\alpha \in I} A_\alpha \right) &= \{ B \in \Omega_1: T(B) \in \bigcup_{\alpha \in I} A_\alpha \} \\
&=\bigcup_{\alpha \in I} \{ B \in \Omega_1: T(B) \in  A_\alpha \} \\
&= \bigcup_{\alpha \in I} T^{-1}(A_\alpha)
\end{align*}
with a similar argument holding for the second. For the third,

\[
\left( T^{-1}(A) \right)^C = \{B \in \Omega_1: T(b) \in A\}^C = \{B \in \Omega_1: T(b) \in A^C\} = T^{-1}(A^C)
\]

\section*{Problem 2.3}

Let $f, g: \Omega \rightarrow \R$ be $\langle \F, \B(\R) \rangle$-measurable. Set

\[
h(\omega) = \frac{f(\omega)}{g(\omega)}\mathds{1}(g(\omega) \neq 0)
\]

Verify that $h$ is $\langle \F, \B(\R) \rangle$-measurable.

Directly from the definition, we have that for $a \in \R$,
\begin{align*}
h^{-1}((-\infty, a]) &= \{\omega: \frac{f(\omega)}{g(\omega)}\mathds{1}(g(\omega) \neq 0) \leq a \} \\
&= \{\omega: f(\omega) \leq ag(\omega), g(\omega) > 0 \} \cup \{\omega: f(\omega) \geq ag(\omega), g(\omega) < 0 \} \\
&= \{\omega: f(\omega) - ag(\omega) \leq 0, g(\omega) > 0 \} \cup \{\omega: f(\omega) - ag(\omega) \geq 0, g(\omega) < 0 \}
\end{align*}

$f(\omega) - ag(\omega)$ and $g(\omega)$ are $\langle \F, \B(\R) \rangle$-measurable functions, so for the left and right sets, the conditions individually define sets in $\F$ so their intersection is in $\F$. Thus $h(\omega)$ is $\langle \F, \B(\R) \rangle$-measurable.

\section*{Problem 2.6}

Let $X_i, i = 1, 2, 3$ be random variables on a probability space $(\Omega, \F, \prob)$. Consider the equation (with $t \in \R$)
\[
X_1(\omega)t^2 + X_w(\omega)t + X_3(\omega) = 0
\]

\subsection*{Part a}
Show that $A \coloneq \{ \omega \in \Omega: \text{The above equation has two distinct roots} \} \in \F$.

The condition for $\omega \in A$ is equivalent to

\[
\{\omega: X_1(\omega) \neq 0 \}\cap \{\omega: X_2^2(\omega) - 4X_1(\omega)X_3(\omega) > 0 \}
\]

because this indicates that the polynomial is second-order and its discriminant is positive. $X_1(\omega)$ and $X_2^2(\omega) = 4X_1(\omega)X_3(\omega)$ are random variables on probability space $(\Omega, \F, \prob)$ and the sets $(0, \infty)$ and $(-\infty, 0) \cup (0, \infty)$ are Borel sets in $\R$, so both of the above sets are in $F$, and their intersection is thus in $F$.

\subsection*{Part b}
Let $T_1(\omega)$ and $T_2(\omega)$ denote the two roots of the above equation on $A$. Let
\[
f_i(\omega) = \begin{cases}
T_i(\omega) & \omega \in A \\
0 & \omega \in A^C
\end{cases}
\]

Show that $(f_1, f_2)$ is $\langle \F, \B(\R^2) \rangle$-measurable.
\begin{lemma}
Let $f$ be a non-negative $\langle \F, \B(\R) \rangle$-measurable function. Then $\sqrt{f}$ is $\langle \F, \B(\R) \rangle$-measurable.
\begin{proof}
For all $a \geq 0$,
\[
(\sqrt{f(\omega)})^{-1}((-\infty, a]) = \{ \omega: 0 \leq \sqrt{f(\omega)} \leq a \} 
= \{\omega : 0 \leq f(\omega) \leq a^2 \}  \in F
\]
\end{proof}
\end{lemma}

By the quadratic formula (arbitrarily letting $i=1$ be the positive root), we have that

\[
f_1(\omega) = \frac{-X_2(\omega) + \sqrt{X_2^2(\omega) - 4X_1(\omega)X_3(\omega)}}{2X_1(\omega)} \mathds{1}_A
\]

and $f_2$ is the negative root. By Problem 2.3, since the numerator and denominator are $\langle \F, \B(\R) \rangle$-measurable, $A \ in F$, and the restriction in $A$ prevents the denominator from being zero, $f_i, i = 1, 2$ is $\langle \F, \B(\R) \rangle$-measurable, and the Cartesian product of $\langle \F, \B(\R) \rangle$-measurable functions is $\langle \F, \B(\R) \rangle$-measurable.

\section*{Problem 2.7}

Let $M \coloneq ((X_{ij})), 1 \leq i,j \leq k$ be a random matrix of random variables $X_{ij}$ defined on a probability space $(\Omega, \F, \prob)$.

\subsection*{Part a}

Show that $det(M)$ and $tr(M)$ are both $\langle \F, \B(\R) \rangle$-measurable.

The trace is trivial; the diagonal entries of $M$ are $\langle \F, \B(\R) \rangle$-measurable and so is their sum. The determinant follows by induction on the size of $M$. When $k=1$, the determinant is the random variable $X_{11}$, which by assumption is $\langle \F, \B(\R) \rangle$-measurable. Assuming that the determinant of a size $k-1$ matrix of random variables in $(\Omega, \F, \prob)$ is $\langle \F, \B(\R) \rangle$-measurable, by the Laplace expansion, we can rewrite the determinant of the size $k$ matrix as the sum of random variables multiplied by the determinant of size $k-1$ matrices, and this sum is $\langle \F, \B(\R) \rangle$-measurable.

\subsection*{Part b}

Show that the largest eigenvalue of $M'M$ is $\langle \F, \B(\R) \rangle$-measurable.

Using the hint, I will note that the largest eigenvalue is equal to

\[
\sup_{x} \frac{x'M'Mx}{x'x}\mathds{1}_{x \neq 0}
\]

The numerator and denominator are the sums of products of random variables in $(\Omega, \F, \prob)$ and the denominator is restricted from zero, so the internal function is $\langle \F, \B(\R) \rangle$-measurable, and the supremum of a measurable function is measurable.

\end{document}