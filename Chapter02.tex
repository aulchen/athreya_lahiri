\documentclass{article}
\usepackage{amsmath}
\usepackage{amsfonts}
\usepackage{amssymb}
\usepackage{optidef}
\usepackage{dsfont}

\newenvironment{proof}{\paragraph{Proof:}}{\hfill$\square$}
\newtheorem{theorem}{Theorem}
\newtheorem{lemma}[theorem]{Lemma}
\newtheorem{corollary}[theorem]{Corollary}

\newcommand{\R}{\mathbb{R}}
\newcommand{\Q}{\mathbb{Q}}
\newcommand{\Z}{\mathbb{Z}}
\newcommand{\N}{\mathbb{N}}

\newcommand{\F}{\mathcal{F}}
\newcommand{\M}{\mathcal{M}}
\newcommand{\B}{\mathcal{B}}
	
\newcommand{\indep}{\perp \!\!\! \perp}

\newcommand{\prob}{\boldsymbol{P}}

\author{Arthur Chen}
\title{Athreya Lahiri Chapter 2 Solutions}
\date{\today}

\begin{document}

\maketitle

\section*{Problem 2.1}

Prove de Morgan's laws. Let $\Omega_i, i = 1, 2$ be two nonempty sets, and let $T:\Omega_1 \rightarrow \Omega_2$ be a map. For any collection $\{A_\alpha: \alpha \in I\}$ of subsets of $\Omega_2$, prove that

\begin{gather*}
T^{-1}\left( \bigcup_{\alpha \in I} A_\alpha \right) = \bigcup_{\alpha \in I} T^{-1}(A_\alpha) \\
T^{-1}\left( \bigcap_{\alpha \in I} A_\alpha \right) = \bigcap_{\alpha \in I} T^{-1}(A_\alpha) \\
\left( T^{-1}(A) \right)^C = T^{-1}(A^C)
\end{gather*}

For the first,
\begin{align*}
T^{-1}\left( \bigcup_{\alpha \in I} A_\alpha \right) &= \{ B \in \Omega_1: T(B) \in \bigcup_{\alpha \in I} A_\alpha \} \\
&=\bigcup_{\alpha \in I} \{ B \in \Omega_1: T(B) \in  A_\alpha \} \\
&= \bigcup_{\alpha \in I} T^{-1}(A_\alpha)
\end{align*}
with a similar argument holding for the second. For the third,

\[
\left( T^{-1}(A) \right)^C = \{B \in \Omega_1: T(b) \in A\}^C = \{B \in \Omega_1: T(b) \in A^C\} = T^{-1}(A^C)
\]

\section*{Problem 2.3}

Let $f, g: \Omega \rightarrow \R$ be $\langle \F, \B(\R) \rangle$-measurable. Set

\[
h(\omega) = \frac{f(\omega)}{g(\omega)}\mathds{1}(g(\omega) \neq 0)
\]

Verify that $h$ is $\langle \F, \B(\R) \rangle$-measurable.

Directly from the definition, we have that for $a \in \R$,
\begin{align*}
h^{-1}((-\infty, a]) &= \{\omega: \frac{f(\omega)}{g(\omega)}\mathds{1}(g(\omega) \neq 0) \leq a \} \\
&= \{\omega: f(\omega) \leq ag(\omega), g(\omega) > 0 \} \cup \{\omega: f(\omega) \geq ag(\omega), g(\omega) < 0 \} \\
&= \{\omega: f(\omega) - ag(\omega) \leq 0, g(\omega) > 0 \} \cup \{\omega: f(\omega) - ag(\omega) \geq 0, g(\omega) < 0 \}
\end{align*}

$f(\omega) - ag(\omega)$ and $g(\omega)$ are $\langle \F, \B(\R) \rangle$-measurable functions, so for the left and right sets, the conditions individually define sets in $\F$ so their intersection is in $\F$. Thus $h(\omega)$ is $\langle \F, \B(\R) \rangle$-measurable.

\section*{Problem 2.6}

Let $X_i, i = 1, 2, 3$ be random variables on a probability space $(\Omega, \F, \prob)$. Consider the equation (with $t \in \R$)
\[
X_1(\omega)t^2 + X_w(\omega)t + X_3(\omega) = 0
\]

\subsection*{Part a}
Show that $A \coloneq \{ \omega \in \Omega: \text{The above equation has two distinct roots} \} \in \F$.

The condition for $\omega \in A$ is equivalent to

\[
\{\omega: X_1(\omega) \neq 0 \}\cap \{\omega: X_2^2(\omega) - 4X_1(\omega)X_3(\omega) > 0 \}
\]

because this indicates that the polynomial is second-order and its discriminant is positive. $X_1(\omega)$ and $X_2^2(\omega) = 4X_1(\omega)X_3(\omega)$ are random variables on probability space $(\Omega, \F, \prob)$ and the sets $(0, \infty)$ and $(-\infty, 0) \cup (0, \infty)$ are Borel sets in $\R$, so both of the above sets are in $F$, and their intersection is thus in $F$.

\subsection*{Part b}
Let $T_1(\omega)$ and $T_2(\omega)$ denote the two roots of the above equation on $A$. Let
\[
f_i(\omega) = \begin{cases}
T_i(\omega) & \omega \in A \\
0 & \omega \in A^C
\end{cases}
\]

Show that $(f_1, f_2)$ is $\langle \F, \B(\R^2) \rangle$-measurable.
\begin{lemma}
Let $f$ be a non-negative $\langle \F, \B(\R) \rangle$-measurable function. Then $\sqrt{f}$ is $\langle \F, \B(\R) \rangle$-measurable.
\begin{proof}
For all $a \geq 0$,
\[
(\sqrt{f(\omega)})^{-1}((-\infty, a]) = \{ \omega: 0 \leq \sqrt{f(\omega)} \leq a \} 
= \{\omega : 0 \leq f(\omega) \leq a^2 \}  \in F
\]
\end{proof}
\end{lemma}

By the quadratic formula (arbitrarily letting $i=1$ be the positive root), we have that

\[
f_1(\omega) = \frac{-X_2(\omega) + \sqrt{X_2^2(\omega) - 4X_1(\omega)X_3(\omega)}}{2X_1(\omega)} \mathds{1}_A
\]

and $f_2$ is the negative root. By Problem 2.3, since the numerator and denominator are $\langle \F, \B(\R) \rangle$-measurable, $A \ in F$, and the restriction in $A$ prevents the denominator from being zero, $f_i, i = 1, 2$ is $\langle \F, \B(\R) \rangle$-measurable, and the Cartesian product of $\langle \F, \B(\R) \rangle$-measurable functions is $\langle \F, \B(\R) \rangle$-measurable.

\section*{Problem 2.7}

Let $M \coloneq ((X_{ij})), 1 \leq i,j \leq k$ be a random matrix of random variables $X_{ij}$ defined on a probability space $(\Omega, \F, \prob)$.

\subsection*{Part a}

Show that $det(M)$ and $tr(M)$ are both $\langle \F, \B(\R) \rangle$-measurable.

The trace is trivial; the diagonal entries of $M$ are $\langle \F, \B(\R) \rangle$-measurable and so is their sum. The determinant follows by induction on the size of $M$. When $k=1$, the determinant is the random variable $X_{11}$, which by assumption is $\langle \F, \B(\R) \rangle$-measurable. Assuming that the determinant of a size $k-1$ matrix of random variables in $(\Omega, \F, \prob)$ is $\langle \F, \B(\R) \rangle$-measurable, by the Laplace expansion, we can rewrite the determinant of the size $k$ matrix as the sum of random variables multiplied by the determinant of size $k-1$ matrices, and this sum is $\langle \F, \B(\R) \rangle$-measurable.

\subsection*{Part b}

Show that the largest eigenvalue of $M'M$ is $\langle \F, \B(\R) \rangle$-measurable.

Using the hint, I will note that the largest eigenvalue is equal to

\[
\sup_{x} \frac{x'M'Mx}{x'x}\mathds{1}_{x \neq 0}
\]

The numerator and denominator are the sums of products of random variables in $(\Omega, \F, \prob)$ and the denominator is restricted from zero, so the internal function is $\langle \F, \B(\R) \rangle$-measurable, and the supremum of a measurable function is measurable.

\section*{Problem 2.8}

Let $f: \R \rightarrow \R$. Let $\overline{f}(x) = \inf_{\delta > 0} \sup_{|y-x|<\delta} f(y)$ and $\underline{f}(x) = \sup_{\delta > 0} \inf_{|y-x|<\delta} f(y), x \in \R$.

\subsection*{Part a}

Show that for any $t \in \R$,

\[
\{ x: \overline{f}(x)<t \}
\]

is open and hence $\overline{f}$ is $\langle \B(\R), \B(\R) \rangle$-measurable.

Denote the set above as $A$. By the definition of $\overline{f}$ and the properties of infimum, for each $x \in A$, there exists $\delta_0$ such that

\[
\sup_{|y-x|<\delta_0} f(y) < t
\]

Thus for all $x'$ such that $|y-x'| < \delta_0$, there exists an open ball centered at $x'$ such that $B(x', r) \subset B(x, \delta_0)$, and thus
\[
\sup_{y \in B(x', r)} f(y) \leq \sup_{y \in B(x, \delta_0)} f(y) < t
\]

implies

\[
\inf_{r \rightarrow 0} \sup_{y \in B(x', r)} f(y) < t
\]

and thus $x' \in A$, which implies that $B(x, \delta_0) \subset A$. Thus $A$ is open. A similar argument holds in reverse for $\underline{f}$ using $\{ x: \underline{f}(x)>t \}$

\subsection*{Part b}

Show that for any $t > 0$,

\[
\{ x: \overline{f}(x) - \underline{f}(x) < t \} = \bigcup_{r \in \Q}\{ x: \overline{f}(x)<t+r, \underline{f}(x) > r \}
\]

and hence is open.

Denote the set on the left $A$. $A$ can be built up as a union of smaller sets. For $r \in \R$, consider $A_r = \{ x: \overline{f}(x)<r+t, \underline{f}(x) = r \}$. It's clear that this captures all $x$ in the inclusion condition of $A$ for a given value of $\underline{f}(x) = r$. By taking the union over all $r \in \R$, we get the inclusion condition of $A$ for all possible $x \in \R$ and thus the union equals $A$. Thus

\[
A = \bigcup_{r \in \R} \{ x: \overline{f}(x)<r+t, \underline{f}(x) = r \}
\] 

Since we're taking the union over all $r \in \R$ and our inclusion criteria is a strictly less than sign, we can replace the $\underline{f}(x) = r$ conditions with $\underline{f}(x) > r$, such that $r \in \Q$ instead of $\R$.

To see this, let $A'_{r} = \{ x: \overline{f}(x)<r+t, \underline{f}(x) > r \}$. If $x \in A_r$, then $\overline{f}(x) < r+t, \underline{f}(x) = r$. Since rationals are dense in the reals and the inequality is strict, there exists $r' \in \Q$ such that $\overline{f}(x)<r'+t, \underline{f}(x) > r'$, and thus $x \in A'_{r'}$. Thus

\[
A = \bigcup_{r \in \Q} \{ x: \overline{f}(x)<r+t, \underline{f}(x) > r \}
\]

But by Part a, each of the subsets is the finite intersection of two open sets and is thus open, and the countable union of open sets is open. Thus $A$ is open, as desired.

\section*{Problem 2.15}

Consider the probability space $\bigl((0, 1), \B((0, 1)), m \bigl)$, where $m$ is the Lebesgue measure.

\subsection*{Part a}

Let $Y_1$ be the random variable $Y_1(x) = \sin(2\pi x)$ for $x \in (0, 1)$. Find the cdf of $Y_1$.

Looking at the graph of $\sin(2\pi x)$, for $y \in (0, 1)$, we have that $\sin^{-1} y > 0$, thus

\begin{align*}
\prob(Y_1 \leq y) &= mY_1^{-1}\bigl( (-\infty, y] \bigr)\\
&= m\left( \left(0, \frac{\sin^{-1}y}{2\pi}\right) \bigcup \left(1/2 - \frac{\sin^{-1}y}{2\pi}, 1\right)  \right)\\
&= \frac{1}{2} + \frac{\sin^{-1}y}{\pi}
\end{align*}

Similarly, for $y \in (-1, 0)$, we have that $\sin^{-1} y < 0$, thus

\[
\prob(Y_1 \leq y) = m\left( \frac{1}{2} - \frac{\sin^{-1}y}{2\pi}, 1 + \frac{\sin^{-1}y}{2\pi} \right) = \frac{1}{2} + \frac{\sin^{-1}y}{\pi}
\]

Thus
\[
\prob(Y_1 \leq y) =
\begin{cases}
0 & y \leq -1 \\
\frac{1}{2} + \frac{\sin^{-1}y}{\pi} & y \in (-1, 1) \\
1 & y \geq 1
\end{cases}
\]

\subsection*{Part b}

Let $Y_2$ be the random variable $Y_2(x) = \log x$ for $x \in (0, 1)$. Find the cdf of $Y_2$.

We have that for $y < 0$,

\[
\prob(Y_2 \leq y) = mY_2^{-1}\bigl( (-\infty, y] \bigl) = m\bigl( (0, e^y) \bigl) = e^y
\]

Thus
\[
\prob(Y_2 \leq y) =
\begin{cases}
e^y & y < 0 \\
1 & y \geq 0
\end{cases}
\]

\subsection*{Part c}

Let $F: \R \rightarrow \R$ be a cdf. For $x \in (0, 1)$, let
\begin{gather*}
F_1^{-1}(x) = \inf\{ y: y \in \R, F(y) \geq x\} \\
F_2^{-1}(x) = \sup\{ y: y \in \R, F(y) \leq x\}
\end{gather*}

Let $Z_i$ be the random variable defined by
\[
Z_i = F_i^{-1}(x), x \in (0, 1), i = 1, 2
\]
\subsubsection*{Subpart i}

\textbf{IN PROGRESS}

Find the cdf of $Z_i, i = 1, 2$.

We begin with a characterization of $F_1^{-1}$ and $F_2^{-1}$.

\begin{lemma}
Let $A_x = \{ y: y \in \R, F(y) = x \}$. Then $A$ is either the empty set, a singleton, or an interval.
\begin{proof}
$F$ is right-continuous and nondecreasing. If there is no $y$ such that $F(y) = x$, then $F$ has a jump discontinuity that jumps from below $y$ to above $y$, since otherwise by the intermediate value theorem $F$ would achieve the value $x$. If $A_x$ is a nonempty nonsingleton, then there exist multiple $y$ such that $F(y) = x$. This must occur when $F$ is flat, and since $F$ is nondecreasing, this can only happen on a connected interval.
\end{proof}
\end{lemma}

\begin{lemma}
$F_i^{-1}(x)$ can be broken up into cases. When $F(y)$ is flat, letting $y_1$ and $y_2$ be the left and right endpoints of the interval, $F_1^{-1}(x) = y_1$ and $F_2^{-1}(x) = y_2$. When $F(y)$ has a jump discontinuity at $y$ that jumps from $x_1$ to $x_2$, $x:x\in[x_1, x_2] \rightarrow F_i^{-1}(x) = y$. Otherwise, $F$ is invertible at $y$ and $F_i^{-1}(x) = F^{-1}(F(y)) = y$.
\end{lemma}


\begin{lemma}
For any $x \in (0, 1), t \in \R, F(t) \geq x \Leftrightarrow F_1^{-1}(x) \leq t$.
\begin{proof}
For the forward, assume that $F(t) \geq x$. Because $F$ is nondecreasing, the infimum of $y$ such that $F(y) \geq x$ must be less than or equal to $t$. For the reverse, assume that $F_1^{-1}(x) \leq t$. Because $F$ is nondecreasing and right continuous, the sets $F(y) \geq x$ are closed intervals, and thus their infimum lies within the set - specifically, at the left endpoint, which is $F_1^{-1}(x)$. Thus for $t \geq F_1^{-1}(x)$, we know that $F(t) \geq F(F_1^{-1}(x)) = x$, since $F_1^{-1}(x)$ is the left endpoint and this is where the sets achieve their minimum.
\end{proof}
\end{lemma}

Thus for $Z_1$,

\[
\prob(Z_1 \leq z) = mZ_1^{-1}\bigl( (-\infty, z] \bigl) = m\bigl( [0, F(z)] \bigl) = F(z)
\]
where we used the lemma substituting $z$ for $t$.

\section*{Problem 2.16}

\subsection*{Part a}

Let $(\Omega, \F_1, \mu)$ be a $\sigma$-finite measure space. Let $T: \Omega \rightarrow \R$ be $\langle \F, \B(\R) \rangle$ measureable. Show by counterexample that the induced measure $\mu T^{-1}$ need not be $\sigma$-finite.

Let $(\Omega, \F_1, \mu)$ be the Lebesgue measure on the Borel sets of $\R$, which is obviously $\sigma$-finite. Let $T(x) = 0$. $T$ is continuous and thus $\langle \F, \B(\R) \rangle$ measurable. However, the induced measure $\mu T^{-1}(A)$ equals infinity if $x \in A$, zero otherwise. Thus there is no collection of sets with finite measure such that their union is $\R$, and thus $\mu T^{-1}$ is not $\sigma$-finite.

\subsection*{Part b}

Let $(\Omega_i, \F_i)$ be measurable spaces for $i = 1, 2$ and let $T: \Omega_1 \rightarrow \Omega_2$ be $\langle \F_1, \F_2 \rangle$-measurable. Show that any measure $\mu$ on $(\Omega_1, \F_1)$ is $\sigma$-finite if $\mu T^{-1}$ is $\sigma$-finite on $(\Omega_2, \F_2)$.

By assumption of $\sigma$-finiteness of $(\Omega_2, \F_2)$, there is a countable collection of sets $\{A_n\}_{n \geq 1} \subset \F_2$ such that $\bigcup_{n \geq 1} A_n = \Omega_2$ and $\mu T^{-1}(A_n) < \infty$ for all $n$. By the first assumption,
\[
\Omega_1 = T^{-1}(\Omega_2) = T^{-1}\left( \bigcup_{n \geq 1} A_n \right) = \bigcup_{n \geq 1} T^{-1}(A_n)
\]

By the second assumption, $\mu(T^{-1}(A_n)) < \infty$ for all $n$. Thus the sets $\{T^{-1}(A_n)\}_{n \geq 1}$ show that $\mu$ is $\sigma$-finite.

\section*{Problem 2.20}

Use Corollary 2.3.5 to show that for any collection $\{a_{ij}: i,j \in \N\}$ of nonnegative numbers,

\[
\sum_{i=1}^\infty \left( \sum_{j=1}^\infty a_{ij} \right) = \sum_{j=1}^\infty \left( \sum_{i=1}^\infty a_{ij} \right)
\]

Let $(\N, P(\N))$ with the counting measure be a measure space, and let $h_n: \N \rightarrow \R$ be the function/sequence $h_i(j) = a_{ij}$ for all $i, j \in \R$. Then

\begin{gather*}
\sum_{i=1}^\infty \int h_i(j) d\mu = \sum_{i=1}^\infty \left( \sum_{j=1}^\infty a_{ij} \right) \\
\int \sum_{i=1}^\infty h_i(j) d\mu = \int \left( \sum_{i=1}^\infty a_{ij} \right) d\mu = \sum_{j=1}^\infty \left( \sum_{i=1}^\infty a_{ij} \right)
\end{gather*}

and by Corollary 2.3.5, since $h_i(j)$ is nonnegative and measurable, the two expressions equal each other.

\end{document}