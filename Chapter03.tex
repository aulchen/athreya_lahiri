\documentclass{article}
\usepackage{amsmath}
\usepackage{amsfonts}
\usepackage{amssymb}
\usepackage{optidef}
\usepackage{dsfont}

\newenvironment{proof}{\paragraph{Proof:}}{\hfill$\square$}
\newtheorem{theorem}{Theorem}
\newtheorem{lemma}[theorem]{Lemma}
\newtheorem{corollary}[theorem]{Corollary}

\newcommand{\R}{\mathbb{R}}
\newcommand{\Q}{\mathbb{Q}}
\newcommand{\Z}{\mathbb{Z}}
\newcommand{\N}{\mathbb{N}}

\newcommand{\F}{\mathcal{F}}
\newcommand{\M}{\mathcal{M}}
\newcommand{\B}{\mathcal{B}}
	
\newcommand{\indep}{\perp \!\!\! \perp}

\newcommand{\prob}{\boldsymbol{P}}

\author{Arthur Chen}
\title{Athreya Lahiri Chapter 3 Solutions}
\date{\today}

\begin{document}

\maketitle

\section*{Problem 3.1}

Let $\phi:(a, b) \rightarrow \R$ be convex. Show the following.

We will constantly use the following convexity formula: if $\phi$ is convex and $a < x_1 < x_2 < x_3 < b$, then 
\[
\frac{\phi(x_2)-\phi(x_1)}{x_2 - x_1} \leq \frac{\phi(x_3)-\phi(x_1)}{x_3 - x_1} \leq \frac{\phi(x_3)-\phi(x_2)}{x_3 - x_2}
\]

\subsection*{Part a}

For each $x \in (a, b)$,

\[
\phi'_+(x) = \lim_{y \downarrow x} \frac{\phi(y) - \phi(x)}{y-x}, \phi'_-(x) = \lim_{y \uparrow x} \frac{\phi(y) - \phi(x)}{y-x}
\]

exist and are finite.

Consider $\phi'_+$ first. Let $z < x$ be arbitrary and let $\{y_n\}_{n = 1} \geq x$ such that $y_n \rightarrow x$. Wlog, let $y_n$ be decreasing. Then by convexity,

\[
\frac{\phi(y_1) - \phi(x)}{y_1-x} \geq \frac{\phi(y_2) - \phi(x)}{y_2-x} \geq \dots
\]

which is bounded below by $\frac{\phi(x) - \phi(z)}{x-z}$. Thus $\frac{\phi(y_n) - \phi(x)}{y_n-x}$ is a monotonically decreasing sequence bounded below, thus it has a limit. Since this sequence was arbitrary, the limit and thus $\phi'_+$ exists. The same applies in reverse for $\phi'+$.

\section*{Problem 3.3}

Prove the following.

\subsection*{Part a}

Let $a_1 \dots a_k$ be real and $p_1 \dots p_k$ be positive numbers such that $\sum_{i=1}^kp_i = 1$. Then

\[
\sum_{i=1}^kp_i\exp(a_i) \geq \exp\left( \sum_{i=1}^kp_ia_i \right)
\]

Let $P$ be the probability measure on $\R$ that assigns probability $p_i$ to point $a_i$ and apply Jensen's inequality with $\phi(x) = e^x$.

\subsection*{Part b}

Let $b_1 \dots b_k$ be nonnegative numbers and $p_1 \dots p_k$ be as in Part a. Then

\[
\sum_{i=1}^k p_ib_i \geq \prod_{i=1}^k b_i^{p_i}
\]

Furthermore, equality holds iff $b_1 = b_2 = \dots = b_k$.

Let $a_i = \log b_i$ and apply Part a. For the iff, since the exponential function is strictly convex, inequality holds iff $f(\omega)$ is a constant, which in this context means that all the $b_i$s are equal.

\subsection*{Part c}

For any $a, b \in \R$ and $1 \leq p < \infty$,

\[
|a+b|^p \leq 2^{p-1}(|a|^p + |b|^p)
\]

Let $f(x) = x$, $\phi(x) = |x|^p$, which is convex on the range of $p$, and let $P$ be the probability measure with 1/2 probability on $\{a, b\}$. Thus by Jensen's inequality,

\[
\phi\left( \int xdP \right) = \frac{1}{2^p}|a+b|^p \leq \int |x|^pdP = \frac{1}{2}(|a|^p + |b|^p)
\]

which implies the desired result.

\section*{Problem 3.12}

\subsection*{Part b}

Prove that for $p \in (0, 1)$, $\int |f+g|^pd\mu \leq \int|f|^pd\mu + \int|g|^pd\mu$.

Building off of equation 2.2, we have that

\[
\biggl( \frac{|x|}{|x|+|y|} \biggr)^p + \biggl( \frac{|y|}{|x|+|y|} \biggr)^p \geq \frac{|x|}{|x|+|y|} + \frac{|y|}{|x|+|y|} = 1
\]

implies

\[
|x+y|^p \leq (|x|+|y|)^p \leq |x|^p + |y|^p
\]

and integrating pointwise gives the result.

\section*{Problem 3.14}

Show that $(L^\infty(\mu), d_\infty)$ is a complete metric space.

Using the hint, let $\{f_n\}_{n \geq 1}$ be a Cauchy sequence in $L^\infty(\mu)$. For each $k \geq 1$, let $f_{n_k}$ be a subsequence such that $||f_{n_{k+1}} - f_{n_k}||_\infty < 2^{-k}$. By the definition of $L^\infty$, for all $f \in L^\infty$, the set $\{\omega: |f(\omega)| > ||f||_\infty\}$ has measure zero.

Let
\[
A = \bigcap_{k=1}^\infty\{ \omega: |f_{n_{k+1}}(\omega) - f_{n_k}(\omega)| \leq ||f_{n_{k+1}} - f_{n_k}||_\infty \}
\]

$A^C$ has measure zero because it is the countable union of zero sets.

Thus, for $\omega \in A$ and for all $k \geq 1$, $|f_{n_{k+1}}(\omega) - f_{n_k}(\omega)| \leq ||f_{n_{k+1}} - f_{n_k}||_\infty < 2^{-k}$. Thus for $\omega \in A$, $\{f_{n_k}(\omega)\}_{k \geq 1}$ is a Cauchy sequence in $\R$, which converges to a point we denote $f(\omega)$. For $\omega \in A^C$, let $f(\omega) = 0$. Then $\lim_{k \rightarrow \infty} f_{n_k} = f$ a.e. $(\mu)$, and the rest of the proof follows as in the proof of Theorem 3.2.2, which proves completeness for $L^p$, $p \in (0, \infty)$.

\end{document}